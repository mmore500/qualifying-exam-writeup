\section{Discussion}

Addressing credit assignment ambiguity is central to evolving genetically heterogeneous groups.
Gomes et al., try to solve credit assignment by guaranteeing evaluation with a good cooperating partner;
thus, responsibility for the partnership's success or failure rests with the evaluated individual.
The process requires jump start; where cooperation is unlikely to TODO
Waibel et al., remove credit assignment ambiguity by directly rewarding individuals for their own contributions to group foraging instead of sharing group foraging rewards evenly between group members.
This approach works well, but in principle cannot account for potentially important indirect contributions to group foraging success like guiding other agents to food pucks.
Knudson et al., suggest the more generalized notion of the difference metric, where individual fitness is based how much the individual's presence in the group increases group performance.
Unless shortcuts to figure individual contributions can be devised, however, this approach might be inefficient because it would require many re-evaluations of the collective, each sans a particular individual member.
The particular shortcut employed by Knudson et al. works analogously to the direct large-puck credit assignment demonstrated by Waibel et al.
Thus, this approach faces a trade-off between efficiency and accounting for indirect contributions to group performance.

Another approach to iron out credit-assignment ambiguity, not directly addressed by the papers reviewed, is to establish a ``team genome,'' essentially redefining the evolutionary unit of individuality as heterogeneous ensembles of agents \cite{miconi2003evolving}.
Tying together the evolutionary fates of team members helps address the problem of inefficient selection observed in heterogeneous-team, group-selection experiments in \cite{waibel2009genetic};
poorly-performing agents can no longer drift from group to group, piggybacking off of the competence of the other agents.
However, as evident in \cite{waibel2009genetic}, genetically homogeneous groups can sometimes out-evolve such heterogeneous groups because of the much larger dimensionality the heterogeneous behavior space.
The clonal generation of a homogeneous group from a single genome might be interpreted as an indirect genetic encoding that exploits symmetries \cite{clune2011performance}.
An indirect encoding that does not enforce completely identical agent configuration but instead allows for subsets of the team to be configured identically or some (but not all) elements of agent configuration to be determined uniformly among team members might ameliorate this issue \cite{bongard2000legion}.

Problem-specific knowledge is a common theme of strategies to thwart credit assignment problems.
Although such contributions aren't relevant to the particular multi-agent problems studied, the methods presented by Waibel et al. and Knudson et al. would be unable to account for indirect contributions to group performance (i.e., contributions beyond towing back large food pucks or proximity to points of interest).
Gomes et al. restrict cooperating groups to pairs of agents by co-evolving two populations.
In larger multi-agent problems, it might be unknown beforehand how many agents --- and what combinations of robotic hardware --- are appropriate for particular task.
Unless proper care is taken \cite{bongard2000legion}, a naive implementation of the ``team genome'' approach might enforce similar restrictions on group size and composition.
Obviously, techniques that presuppose no knowledge of the composition, size, and mechanisms of multi-agent teams are more convenient to use.
In more ambitious or open-ended applications of multi-agent team evolution, such knowledge might simply be unavailable \cite{kernbach2008symbiotic,baele2009open}.
Future work on evolving genetically-heterogeneous cooperating might look for further inspiration from nature, where cooperative relationships between members of distant branches the tree of life abound by means of phenomena like transitions of individuality, vertical transmission, ecological interactions, partner selection, and reciprocity \cite{vostinar2017suicide, andre2016evolutionary}.
