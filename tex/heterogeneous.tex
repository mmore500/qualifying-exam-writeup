\section{Multi-Agent Systems}

The term multi-agent system (MAS) describes a scenario where a behavior manifests, or in more application-driven context a goal is accomplished, via a collection of interacting autonomous entities \cite{ferber2003agents}.
An ant colony is a classic example of a multi-agent system.
Through many interactions between individual ants (including those mediated by stigmergic mechanisms), the colony arrives at collective decisions --- for example, discerning the richer of two food sources \cite{beckers1993modulation} or depositing a certain proportion of the workforce into living bridges that shortcut obstacles \cite{graham2017optimal} --- that optimize the inflow of foraged food.
Computational and mathematical models of this system are  put to use [cite]
Can be applied more abstractly, for example, decomposing a power grid in order to achieve efficient policies for power distribution [cite].

\section{Heterogeneity}

Interesting multi-agent systems are heterogeneous in some manner.
Something like standing in between two mirrors.
A completely homogeneous multi-agent system


examples of multi-agent systems.


Symmetry breaking, initial conditions, state. \cite{banda2015configuration} \cite{boldi1996symmetry} \cite{angluin1980local}

In the context of evolutionary algorithms using genetic programming to control systems of agents, we'll focus on heterogeneousness in terms beyond transient state: genetic heterogeneousness.

\section{Genetic Heterogeneity}

what is genetic het?

why genetic het? : specialization
+ different behaviors (biological analogy; ex shepherd fox sheep \cite{potter2001heterogeneity} )
  + can also be accomplished by state heterogeneousness (genetic differentiation/using different parts of genome) \cite{ferrante2015evolution}
+ different capabilities/hardware of the agents \cite{mathews2012supervised}
  + example?
  + proof-of-concept work shows can be accomplished also through plasticity, \cite{tuci2008evolving}
  + variation in unit-to-unit specs \cite{pugh2007parallel} \cite{duarte2016evolution}

blurred line between individuality and het
