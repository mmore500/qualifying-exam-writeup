\section{Coevolution of heterogeneous multi-robot teams \cite{knudson2010coevolution}}

Knudson et al. focus on a multi-rover scouting task, which seems to have been chosen in part because it requires coordination between individual robots while allowing for relatively straightforward assessment of the impact of an individual robot on overall team performance.
The task centers around the close-range observation of a set of points of interest, a scenario with parallels to, among other applications, hunting for resources on extraterrestrial surfaces like Mars.
A close-range observation is considered to have occurred if a robot passes within a threshold distance of the point of interest.
Coordination comes into play because certain goldilocks numbers of robots must observe a point; too few, and the point isn't characterized in adequate detail (or with adequate certainty), too many, and rover time --- which could be applied to characterizing other points of interest --- is wasted.
This with absolute step-wise cutoffs and more continuous slopes on payout
Most of these points remain static between evaluations, but a subset of points change in order to discourage behavioral overfitting.

The introduce blue and green robots, which have complementary sensing capabilities and for a point to be adequately characterized must be visited by both types of robots.
