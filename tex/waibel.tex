\section{Genetic team composition and level of selection in the evolution of cooperation \cite{waibel2009genetic}}

Waibel et al. study a simple ten-agent foraging problem designed to allow straightforward assessment --- and experimental manipulation --- of the amount of cooperation required.
To collect food, agents pushed weighted tokens to a certain wall of a fully-enclosed rectangular arena.
Two token types were considered: small tokens, which could be transported by a single agent, and large tokens, which were too heavy for a single agent but which could be transported by joint effort between two agents.
Three foraging task variants were studied:
\begin{enumerate}
\item Individual Foraging, where only small tokens were present and fitness was determined by an agent's own foraging contributions (i.e., the number of tokens it pushed to the home wall), so coordinated efforts were not required and payoff was not shared;
\item Cooperative Foraging, where only large tokens were present and fitness was determined by the total token foraging of the collective; and
\item Altruistic Cooperative Foraging, where large tokens, with benefits shared among the entire collective, and small tokens, with benefits restricted to each token's forager, were available.
\end{enumerate}

Agents take the form of simulated small two-wheeled robots.
Although the simulated robots were based on available physical robotic units (pictured in the paper), all reported experimental results were attained exclusively in simulation for efficiency's sake.
These robots have basic distance-sensing and vision capabilities but lack any means for direct robot-to-robot communication.
Control for the robots was accomplished through a fully-connected feed-forward artificial neural network with a single three-node hidden layer.
Neural net output controls rotation of the left and right motors.
A fixed neural network topology was enforced; only neural net weights evolved.

Waibel et al. compare the evolution of genetically homogeneous versus heterogeneous teams under individual versus team selection.
In heterogeneous team treatments, teams were populated member-by-member through fitness-weighted random draws of members from the previous generation.
In homogeneous team treatments, teams were populated through clonal multiplication of a single fitness-weighted random draw from members of the previous generation.
Under team selection, fitness-weighted random draws were conducted by first performing a roulette-wheel draw among previous-generation teams (each team's probability of selection proportional to the mean fitness of its members) followed by an evenly-weighted random draw among the selected team's members.
Under team selection, fitness-weighted random draws were performed from all individuals in the previous generation, with each individual's probability of selection proportional to its individual fitness.

For the Individual Foraging task, individual and team selection on homogeneous teams was exactly equivalent under this particular implementation, and, as expected, yielded champion teams with statistically indistinguishable net foraging performance after 300 generations.
After adjustments to take into account the greater number of distinct genomes evaluated when non-clonal groups are employed, champion performance under heterogeneous team individual selection was also found to be indistinguishable compared to team and individual selection with homogeneous teams.
However, team selection with heterogeneous team make-up yielded much lower champion performance.
Waibel et al. suggest that selection inefficiencies were to blame; simply put, high quality individual genomes are more difficult to detect and select for (and low quality individual genomes are more difficult to detect and select against) when selection only takes place for or against arbitrary lumped-together sets of genomes (which likely contain both high-quality and low-quality solutions);
good quality genomes still \textit{tend} to leave more offspring than poor quality genomes because they \textit{tend} to be part of better teams by means of their own quality, but the selective signal is nonetheless weakened.
These results show that credit assignment problems can emerge even in the absence of direct agent-to-agent cooperation, causing noisier selection.

For the Cooperative Foraging Task, homogeneous teams significantly outperformed heterogeneous teams under both the individual and team selection treatments.
(The individual and team selection treatments were functionally identical in this case, as the large token haul was shared evenly among team members in either case).
Next, the authors test the hypothesis that credit assignment issues caused the poorer performance of heterogeneous teams by performing a heterogeneous individual-selection run where fitness points from each large token were directly allocated to the two robots that collected it.
Performance of heterogeneous teams increased with removal of credit-assignment ambiguities, but still lagged behind the performance of homogeneous teams.
Waibel et al. suggest that evolution of heterogeneous teams lagged behind evolution of homogeneous teams because the search space of possible heterogeneous cooperative behaviors outstrips the search space of possible homogeneous team behaviors; when a unique controller operates each agent many more configurations are possible, so finding a good solution --- workable configurations for \textit{all} team members --- becomes more difficult.
Selection inefficiencies due to credit assignment problems were therefore partly, but not completely, to blame for the poor performance of heterogeneous teams; evolving heterogeneous teams also simply poses a more complex optimization challenge.

The credit assignment problem emerged once again in the Altruistic Cooperative Foraging Task.
In this task, two options were presented to agents:
\begin{itemize}
\item cooperate in work that benefits the group (i.e., collecting large tokens with fitness points shared among group members), or
\item defect to perform tasks that benefit oneself (i.e., collecting small tokens with fitness points assigned to the collector).
\end{itemize}
Every individual agent stands to make out the best when collecting small tokens for itself while accepting a share of the fitness points brought in from harvest of large tokens by \textit{other} agents.
However, the collective collects the least food overall when all agents focus on collecting small tokens for themselves.
So, under individual selection, selective pressure occurs for defection and evolved team performance is poor.
Team selection solves the issue of incentivized defection, but suffers from the same selective inefficiency (i.e., cannot detect and select for/against good /bad genomes that are grouped onto the same team) touched on earlier.
Waibel demonstrate that, with individual selection on heterogeneous groups, assigning points for large token collection to the responsible pair of agents solves the credit assignment issue causing defection under individual selection, showing that preferential foraging for large tokens evolves under these conditions.
Reformulating fitness scoring to directly reward agents for cooperative actions that benefit the collective (like foraging big tokens) can skirt the credit assignment problem.
However, it is possible to imagine scenarios --- like scouting in a multi-agent foraging task --- where such reformulations to directly credit heterogeneous agents for the utility of their behavior may not be practical, or even possible (is the scout still performing an important task even if it never detects any food objects for other agents to exploit? does the answer depend on how many other agents are scouting?).
