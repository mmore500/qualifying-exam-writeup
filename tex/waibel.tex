\section{Genetic team composition and level of selection in the evolution of cooperation \cite{waibel2009genetic}}

Waibel et al. study a simple ten-agent foraging problem designed to allow straightforward assessment --- and experimental manipulation --- of the amount of cooperation required.
To collect food, agents pushed weighted tokens to a certain wall of a fully-enclosed rectangular arena.
Two token types were considered: small tokens, which could be transported by a single agent, and large tokens, which were too heavy for a single agent but which could be transported by joint effort between two agents.
Three foraging task variants were studied:
\begin{enumerate}
\item Individual Foraging, where only small tokens were present and fitness was determined by an agent's own foraging contributions (i.e., the number of tokens it pushed to the home wall), so coordinated efforts were not required and payoff was not shared;
\item Cooperative Foraging, where only large tokens were present and fitness was determined by the total token foraging of the collective; and
\item Altruistic Cooperative Foraging, where large tokens, with benefits shared among the entire collective, and small tokens, with benefits restricted to each token's forager, were available.
\end{enumerate}

Agents take the form of simulated small two-wheeled robots.
Although the simulated robots were based on available physical robotic units (pictured in the paper), all reported experimental results were attained exclusively in simulation for efficiency's sake.
These robots have basic distance-sensing and vision capabilities but lack any means for direct robot-to-robot communication.
Control for the robots was accomplished through a fully-connected feed-forward artificial neural network with a single three-node hidden layer.
Neural net output controls rotation of the left and right motors.
A fixed neural network topology is enforced; only neural net weights evolve.

Waible et al. compare the evolution of genetically homogenous versus heterogenous teams with individual versus team selection.
In heterogneous team treatments, teams were populated member-by-member through fitness-weighted random draws of members from the previous generation.
In homogenous team treatments, teams were populated through clonal multiplication of a single fitness-weighted random draw from members of the previous generation.
Under team selection, fitness-weighted random draws were conducted by first performing a roulette-wheel draw among previous-generation teams (each team's probability of selection proportional to the mean fitness of its members) followed by an evenly-weighted random draw among the selected team's members.
Under team selection, fitness-weighted random draws were performed from all individuals in the previous generation, with each individual's probability of selection proportional to its individual fitness.

For the Individual Foraging task, individual and team selection on homogenous teams was exactly equivalent under this particular implementation, and, as expected, yielded champion teams with statistically indistinguishable net foraging performance after 300 generations.
After adjustments to take into account the greater number of distinct genomes evaluated when non-clonal groups are employed, champion performance under heterogenous team individual selection was also found to be indistinguishable compared to team and individual selection with homogenous teams.
However, team selection on heterogenous teams yielded much lower champion performance.
Credit assignment problems even when cooperation not involved TODO

For the Cooperative Foraging Task

Finally, for the Altruistic Cooperative Foraging Task,
